%% !Mode:: "TeX:UTF-8"
\documentclass{mcmthesis}
%  \documentclass[CTeX = true]{mcmthesis}
\mcmsetup{tstyle=\color{red}\bfseries,%修改题号,队号的颜色和加粗显示,黑色可以修改为 black
        tcn = 2512625, problem = C, %修改队号,参赛题号
        sheet = true, titleinsheet = true, keywordsinsheet = true,
        titlepage = false, abstract = true}

% 代码块设置
    \usepackage{fontspec}
    \setmainfont{Charter} [
        Extension = .ttf,
        UprightFont = * Regular,
        BoldFont = * Bold, 
        ItalicFont = * Italic,
        BoldItalicFont = * Bold Italic
    ]
    \newfontfamily\consolasfont{Consola}[
        Extension = .ttf,
        UprightFont = *,
        BoldFont = *b,
        ItalicFont = *i,
        BoldItalicFont = *z
    ]

    \definecolor{bg}{rgb}{0.92,0.95,1.0} % Lighter blue fill
    \definecolor{borderblue}{rgb}{0.4,0.4,1.0} % Blue border
    \definecolor{commentcolor}{rgb}{0.4,0.85,0.4} % Softer green comment color

    \lstset{
        basicstyle=\consolasfont,  % Changed to use Consolas
        backgroundcolor=\color{bg},
        lineskip=1.5pt,
        frame=single,
        framesep=1mm,
        rulecolor=\color{borderblue}, % Blue border
        numbers=left,
        numberstyle=\small\color{gray}, % Changed from \tiny to \small
        keywordstyle=\color{blue},
        commentstyle=\color{green},
        commentstyle=\color{commentcolor}\itshape,
        stringstyle=\color{red},
        showstringspaces=false,
        breaklines=true,  % Ensure lines break within margins
        breakatwhitespace=true, % Allow breaking at whitespace
        prebreak=\raisebox{0ex}[0ex][0ex]{\ensuremath{\hookleftarrow}},
        postbreak=\raisebox{0ex}[0ex][0ex]{\ensuremath{\hookrightarrow}},
        escapeinside={(*@}{@*)},
        literate={-}{{\textendash}}1
    }
 
\usepackage{indentfirst}  %首行缩进,注释掉,首行就不再缩进。
\usepackage{lipsum}
\title{The \LaTeX{} Template for MCM Version \MCMversion}
\author{\small \href{https://www.latexstudio.net/}
  {\includegraphics[width=7cm]{mcmthesis-logo}}}
\date{\today}

\begin{document}
\begin{abstract}
    \par 
        SUMMARY HERE

\begin{keywords}
    keyword1; keyword2
\end{keywords}

\end{abstract}
\maketitle
\tableofcontents
\newpage

\section{Introduction}
\subsection{...}

\begin{Theorem} \label{thm:latex}
    \LaTeX
\end{Theorem}

\begin{Lemma} \label{thm:tex}
    \TeX .
\end{Lemma}

\begin{proof}
    The proof of theorem.
\end{proof}

\subsection{Other Assumptions}


\section{Analysis of the Problem}

\section{Calculating and Simplifying the Model}

\section{The Model Results}

\section{Validating the Model}

\section{Conclusions}

\section{A Summary}

\section{Evaluate of the Model}

\section{Strengths and weaknesses}

\subsection{Strengths}

\subsection{How to cite?}
bibliography cite use \cite{1}

AI cite use \AIcite{AI1,AI2,AI3}

\begin{thebibliography}{99}
\bibitem{1} ...
\end{thebibliography}

\begin{appendices}

\section{First appendix}

% In addition, your report must include a letter to the Chief Financial Officer (CFO) of the Goodgrant Foundation, Mr. Alpha Chiang, that describes the optimal investment strategy, your modeling approach and major results, and a brief discussion of your proposed concept of a return-on-investment (ROI). This letter should be no more than two pages in length.

\begin{letter}{Dear, Mr. Alpha Chiang}

    \vspace{\parskip}

    Sincerely yours,

    Your friends

\end{letter}

\section{Second appendix}

    % \lstinputlisting[language=C++]{./code/mcmthesis-sudoku.cpp}

\end{appendices}

% AI report begins here
\AImatter
\begin{ReportAiUse}{9}
\bibitem{AI1}



\end{ReportAiUse}

\end{document}
