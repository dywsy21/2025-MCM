%% !Mode:: "TeX:UTF-8"
\documentclass{mcmthesis}
%  \documentclass[CTeX = true]{mcmthesis}
\mcmsetup{tstyle=\color{red}\bfseries,%修改题号,队号的颜色和加粗显示,黑色可以修改为 black
        tcn = 2512625, problem = C, %修改队号,参赛题号
        sheet = true, titleinsheet = true, keywordsinsheet = true,
        titlepage = false, abstract = true}

% 代码块设置
    \usepackage{fontspec}
    \setmainfont{Charter} [
        Extension = .ttf,
        UprightFont = * Regular,
        BoldFont = * Bold, 
        ItalicFont = * Italic,
        BoldItalicFont = * Bold Italic
    ]
    \newfontfamily\consolasfont{Consola}[
        Extension = .ttf,
        UprightFont = *,
        BoldFont = *b,
        ItalicFont = *i,
        BoldItalicFont = *z
    ]

    \definecolor{bg}{rgb}{0.92,0.95,1.0} % Lighter blue fill
    \definecolor{borderblue}{rgb}{0.4,0.4,1.0} % Blue border
    \definecolor{commentcolor}{rgb}{0.4,0.85,0.4} % Softer green comment color

    \lstset{
        basicstyle=\consolasfont,  % Changed to use Consolas
        backgroundcolor=\color{bg},
        lineskip=1.5pt,
        frame=single,
        framesep=1mm,
        rulecolor=\color{borderblue}, % Blue border
        numbers=left,
        numberstyle=\small\color{gray}, % Changed from \tiny to \small
        keywordstyle=\color{blue},
        commentstyle=\color{green},
        commentstyle=\color{commentcolor}\itshape,
        stringstyle=\color{red},
        showstringspaces=false,
        breaklines=true,  % Ensure lines break within margins
        breakatwhitespace=true, % Allow breaking at whitespace
        prebreak=\raisebox{0ex}[0ex][0ex]{\ensuremath{\hookleftarrow}},
        postbreak=\raisebox{0ex}[0ex][0ex]{\ensuremath{\hookrightarrow}},
        escapeinside={(*@}{@*)},
        literate={-}{{\textendash}}1
    }
 
\usepackage{indentfirst}  %首行缩进,注释掉,首行就不再缩进。
\usepackage{lipsum}
\title{Mathematical Model for Prediction of Olympic Medal Counts}
\author{\small \href{https://www.latexstudio.net/}
  {\includegraphics[width=7cm]{mcmthesis-logo}}}
\date{\today}

\begin{document}
\begin{abstract}
    \par 
        SUMMARY HERE

\begin{keywords}
    keyword1; keyword2
\end{keywords}

\end{abstract}
\maketitle
\tableofcontents
\newpage

\section{Introduction}
\subsection{Problem Background}
The Olympic medal table is a key focus for nations and fans, reflecting athletic success and national pride. Predicting medal counts, however, is challenging due to the complex factors involved, such as event types, host country advantages, and the emergence of new competitors. This problem requires developing models exclusively based on the provided datasets, including historical medal tables, event breakdowns, and athlete performance.

Traditional forecasting methods, like OLS regression and Poisson models, struggle with accuracy, particularly for countries with zero or few medals. This problem emphasizes predicting medal breakthroughs for such nations, which demands innovative approaches beyond historical trends.

Key aspects include:
\begin{enumerate}
    \item Exploring the relationship between events and medal distributions.
    \item Examining host country advantages and their influence on results.
    \item Assessing the impact of "great coaches" on medal performance, identifying sports where targeted investment in coaching could make a difference.
    \item Projecting medal counts for the 2028 Los Angeles Olympics, including prediction intervals, while addressing uncertainty and potential breakthroughs for nations without prior Olympic success.
\end{enumerate}

\begin{Theorem} \label{thm:latex}
    \LaTeX
\end{Theorem}

\begin{Lemma} \label{thm:tex}
    \TeX .
\end{Lemma}

\begin{proof}
    The proof of theorem.
\end{proof}

\subsection{Other Assumptions}


\section{Assumptions and Justifications}

\section{Notations and Definitions}


\section{Data Preprocessing}

\section{Establishing the Model}

\section{Task1}

\section{Task2}

\section{Task3}

\section{Sensitivity Analysis}

\subsection{Strengths}

\subsection{How to cite?}
bibliography cite use \cite{1}

AI cite use \AIcite{AI1,AI2,AI3}

\begin{thebibliography}{99}
\bibitem{1} Schlembach C, Schmidt S L, Schreyer D, et al. Forecasting the Olympic medal distribution–a socioeconomic machine learning model[J]. Technological Forecasting and Social Change, 2022, 175: 121314.


\end{thebibliography}

\begin{appendices}

\section{First appendix}

% In addition, your report must include a letter to the Chief Financial Officer (CFO) of the Goodgrant Foundation, Mr. Alpha Chiang, that describes the optimal investment strategy, your modeling approach and major results, and a brief discussion of your proposed concept of a return-on-investment (ROI). This letter should be no more than two pages in length.

\begin{letter}{Dear, Mr. Alpha Chiang}

    \vspace{\parskip}

    Sincerely yours,

    Your friends

\end{letter}

\section{Second appendix}

    % \lstinputlisting[language=C++]{./code/mcmthesis-sudoku.cpp}

\end{appendices}

% AI report begins here
\AImatter
\begin{ReportAiUse}{9}
\bibitem{AI1}



\end{ReportAiUse}

\end{document}
