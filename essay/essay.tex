\documentclass[12pt]{article}
\usepackage{geometry}
\geometry{left=1in,right=0.75in,top=1in,bottom=1in}

\newcommand{\Problem}{$ABCDEFG$}
\newcommand{\Team}{$team$}

\usepackage{ebgaramond}
\usepackage{amsmath,amssymb,amsthm}
% \usepackage{garamondx-math} % must come after amsXXX

\usepackage{graphicx}
\usepackage{xcolor}
\usepackage{fancyhdr}
\usepackage{shellesc}
\lhead{Team \Team}
\rhead{}
\cfoot{}

\newfontfamily\consolasfont{Consola}[
    Extension = .ttf,
    UprightFont = *,
    BoldFont = *b,
    ItalicFont = *i,
    BoldItalicFont = *z
]

\usepackage{listings}

\definecolor{bg}{rgb}{0.92,0.95,1.0} % Lighter blue fill
\definecolor{borderblue}{rgb}{0.4,0.4,1.0} % Blue border
\definecolor{commentcolor}{rgb}{0.4,0.85,0.4} % Softer green comment color


\lstset{
    basicstyle=\consolasfont,  % Changed to use Consolas
    backgroundcolor=\color{bg},
    frame=single,
    framesep=1mm,
    rulecolor=\color{borderblue}, % Blue border
    numbers=left,
    numberstyle=\small\color{gray}, % Changed from \tiny to \small
    keywordstyle=\color{blue},
    % commentstyle=\color{green},
    commentstyle=\color{commentcolor}\itshape,
    stringstyle=\color{red},
    showstringspaces=false,
    breaklines=true,  % Ensure lines break within margins
    breakatwhitespace=true, % Allow breaking at whitespace
    prebreak=\raisebox{0ex}[0ex][0ex]{\ensuremath{\hookleftarrow}},
    postbreak=\raisebox{0ex}[0ex][0ex]{\ensuremath{\hookrightarrow}},
    escapeinside={(*@}{@*)},
    literate={-}{{\textendash}}1
}


\newtheorem{theorem}{Theorem}
\newtheorem{corollary}[theorem]{Corollary}
\newtheorem{lemma}[theorem]{Lemma}
\newtheorem{definition}{Definition}

%%%%%%%%%%%%%%%%%%%%%%%%%%%%%%%%
\begin{document}
\thispagestyle{empty}
\vspace*{-16ex}
\centerline{\begin{tabular}{*3{c}}
	\parbox[t]{0.3\linewidth}{\begin{center}\textbf{Problem Chosen}\\ \Large \textcolor{red}{\Problem}\end{center}}
	& \parbox[t]{0.3\linewidth}{\begin{center}\textbf{2025\\ MCM/ICM\\ Summary Sheet}\end{center}}
	& \parbox[t]{0.3\linewidth}{\begin{center}\textbf{Team Control Number}\\ \Large \textcolor{red}{\Team}\end{center}}	\\
	\hline
\end{tabular}}
%%%%%%%%%%% Begin Summary %%%%%%%%%%%
% Enter your summary here replacing the (red) text
% Replace the text from here ...
\begin{center}
{\Large \textbf{Mathematical Contest in Modeling 2025}} \\[1.5ex]

\textbf{Summary} \\[1.3ex]

\textcolor{black}{%
Use this template to begin typing the first page (summary page) of your electronic report. This 
template uses a 12-point Times New Roman font. Submit your paper as an Adobe PDF 
electronic file (e.g. 1111111.pdf), typed in English, with a readable font of at least 12-point type.	\\[2ex]
Do not include the name of your school, advisor, or team members on this or any page.	\\[2ex]
Be sure to change the control number and problem choice above.	\\
You may delete these instructions as you begin to type your report here. 	\\[2ex]
\textbf{Follow us @COMAPMath on X or COMAPCHINAOFFICIAL on Weibo for the 
most up to date contest information.}
}
\end{center}
% to here
%%%%%%%%%%% End Summary %%%%%%%%%%%

%%%%%%%%%%%%%%%%%%%%%%%%%%%%%%
\clearpage
\pagestyle{fancy}
% Uncomment the next line to generate a Table of Contents
%\tableofcontents
\newpage
\setcounter{page}{1}
\rhead{Page \thepage\ }
%%%%%%%%%%%%%%%%%%%%%%%%%%%%%%
Begin your paper here

$
\exp \left(\left[\begin{array}{llllll}
1 & 0 & 0 & 1 & 0 & 0 \\
0 & 1 & 0 & 0 & 2 & 0 \\
0 & 0 & 2 & 0 & 0 & 3 \\
0 & 0 & 0 & 2 & 0 & 0 \\
0 & 0 & 0 & 0 & 3 & 0 \\
4 & 0 & 0 & 0 & 0 & 3
\end{array}\right]\right)
$
\newline
$
\sum_{k=0}^{\infty}\left(A^*\right)^k A^k
$
\newline

$
\operatorname{det}\left(\frac{A_1+A_2+\cdots+A_m}{m}\right) \geq\left(\operatorname{det}\left(A_1\right) \operatorname{det}\left(A_2\right) \cdots \operatorname{det}\left(A_m\right)\right)^{1 / m}
$

\begin{lemma}
This is a lemma.
\end{lemma}

\begin{theorem}
This is a theorem.
\end{theorem}

\begin{corollary}
This is a corollary.
\end{corollary}

\begin{definition}
This is a definition.
\end{definition}

\begin{lstlisting}[language=Python]
import numpy as np
import matplotlib.pyplot as plt

x = np.linspace(0, 10, 100)

\end{lstlisting}
%%%%%%%%%%%%%%%%%%%%%%%%%%%%%%
\end{document}
