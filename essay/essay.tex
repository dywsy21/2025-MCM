%% !Mode:: "TeX:UTF-8"
\documentclass{mcmthesis}
%  \documentclass[CTeX = true]{mcmthesis}
\mcmsetup{tstyle=\color{red}\bfseries,%修改题号,队号的颜色和加粗显示,黑色可以修改为 black
        tcn = 2512625, problem = C, %修改队号,参赛题号
        sheet = true, titleinsheet = true, keywordsinsheet = true,
        titlepage = false, abstract = true}

% 代码块设置
    \usepackage{fontspec}
    \setmainfont{Charter} [
        Extension = .ttf,
        UprightFont = * Regular,
        BoldFont = * Bold, 
        ItalicFont = * Italic,
        BoldItalicFont = * Bold Italic
    ]
    \newfontfamily\consolasfont{Consola}[
        Extension = .ttf,
        UprightFont = *,
        BoldFont = *b,
        ItalicFont = *i,
        BoldItalicFont = *z
    ]

    \definecolor{bg}{rgb}{0.92,0.95,1.0} % Lighter blue fill
    \definecolor{borderblue}{rgb}{0.4,0.4,1.0} % Blue border
    \definecolor{commentcolor}{rgb}{0.4,0.85,0.4} % Softer green comment color

    \lstset{
        basicstyle=\consolasfont,  % Changed to use Consolas
        backgroundcolor=\color{bg},
        lineskip=1.5pt,
        frame=single,
        framesep=1mm,
        rulecolor=\color{borderblue}, % Blue border
        numbers=left,
        numberstyle=\small\color{gray}, % Changed from \tiny to \small
        keywordstyle=\color{blue},
        commentstyle=\color{green},
        commentstyle=\color{commentcolor}\itshape,
        stringstyle=\color{red},
        showstringspaces=false,
        breaklines=true,  % Ensure lines break within margins
        breakatwhitespace=true, % Allow breaking at whitespace
        prebreak=\raisebox{0ex}[0ex][0ex]{\ensuremath{\hookleftarrow}},
        postbreak=\raisebox{0ex}[0ex][0ex]{\ensuremath{\hookrightarrow}},
        escapeinside={(*@}{@*)},
        literate={-}{{\textendash}}1
    }
 
\usepackage{indentfirst}  %首行缩进,注释掉,首行就不再缩进。
\usepackage{lipsum}
\title{Mathematical Model for Prediction of Olympic Medal Counts}
\author{\small \href{https://www.latexstudio.net/}
  {\includegraphics[width=7cm]{mcmthesis-logo}}}
\date{\today}

\begin{document}
\begin{abstract}
    \par 
        SUMMARY HERE

\begin{keywords}
    keyword1; keyword2
\end{keywords}

\end{abstract}
\maketitle
\tableofcontents
\newpage

\section{Introduction}
\subsection{Problem Background}
The Olympic medal table is a key focus for nations and fans, reflecting athletic success and national pride. Predicting medal counts, however, is challenging due to the complex factors involved, such as event types, host country advantages, and the emergence of new competitors. This problem requires developing models exclusively based on the provided datasets, including historical medal tables, event breakdowns, and athlete performance.

Traditional forecasting methods, like OLS regression and Poisson models, struggle with accuracy, particularly for countries with zero or few medals. This problem emphasizes predicting medal breakthroughs for such nations, which demands innovative approaches beyond historical trends.

Key aspects include:
\begin{enumerate}
    \item Exploring the relationship between events and medal distributions.
    \item Examining host country advantages and their influence on results.
    \item Assessing the impact of "great coaches" on medal performance, identifying sports where targeted investment in coaching could make a difference.
    \item Projecting medal counts for the 2028 Los Angeles Olympics, including prediction intervals, while addressing uncertainty and potential breakthroughs for nations without prior Olympic success.
\end{enumerate}

\subsection{Literature Review}

Predicting Olympic medal counts has been a topic of interest for researchers across various fields, including economics, sociology, and sports science. Early studies, such as those by Ball (1972) and Grimes et al. (1974), \cite{1} focused on identifying fundamental socioeconomic and demographic factors that influence a nation's medal count. These studies emphasized the importance of population size and economic resources in determining Olympic success. Subsequent research, including work by Bernard and Busse (2004) and Xun Bian (2005) \cite{2}, further explored the impact of political systems and hosting advantages on medal counts, finding that hosting the Games and having a centrally-planned economy can significantly enhance a country's performance.

Recent advancements in statistical and machine learning techniques have led to more sophisticated models for predicting Olympic medals. Schlembach et al. (2022) \cite{5} applied a two-staged Random Forest model, incorporating a wide range of socioeconomic variables, including GDP, population, and public health indicators. This approach outperformed traditional models and naïve forecasts, demonstrating the potential of machine learning to improve prediction accuracy. Similarly, Forrest et al. (2010) \cite{4} enhanced traditional regression-based models by including additional regressors such as public spending on recreation and the effects of future hosts. These studies highlight the importance of considering multiple factors, including public health crises like COVID-19, in predicting Olympic performance.

Despite significant progress, predicting Olympic medals remains a complex task due to the interplay of numerous factors. It's recommended that future research should focus on incorporating more granular data, such as individual athlete performance and training conditions, to enhance prediction accuracy. Additionally, the use of panel data and advanced machine learning techniques, such as neural networks and ensemble methods, could provide further insights into the determinants of Olympic success. It's also deemed important to address the exogeneity of the total number of medals available and considering cultural factors, such as political freedom and gender equality. Overall, the field has evolved from simple correlation-based models to sophisticated machine learning algorithms, with traditional socioeconomic factors remaining significant predictors of Olympic success.

\subsection{Our Work}

Due to the limitation of usable dataset, we cannot explore many of the aforementioned factors in detail. Instead, we focus on developing a model that predicts Olympic medal counts based on historical data and sport-specific information. Our model has the following features:

\begin{enumerate}
    \item A two-stage model that first predicts the number of medals won by each country in each sport, then aggregates these predictions to estimate the total medal count.
    \item Trains and evaluates multiple models with different underlying algorithms on the same dataset, and use the best-performing model for prediction.
    \item Separates different countries into multiple clusters based on their historical medal counts (in which their comprehensive national power is also embodied \cite{7}), allowing for more accurate predictions for countries with similar overall strengths.
    \item A sensitivity analysis that assesses the impact of different factors on medal counts and identifies key drivers of success. 
\end{enumerate}

\subsection{Other Assumptions}


\section{Assumptions and Justifications}

\section{Notations and Definitions}


\section{Data Preprocessing}
% remember to mention some details
% 1. event/sport name changed over the course of time
% 2. some special years like 1906, when an unofficial Olympic Games was held. We need to exclude this year from our analysis.
% 3. event name ambiguity, e.g. 500+ events named "Men" and "Women"




\section{Establishing the Model}

\section{Task1 --- Predicting Medal Counts for the 2028 Los Angeles Olympics}

\section{Task2 --- }

\section{Task3}

\section{Task4 --- The Great Coach Effect: its identification and impact}

The Great Coach Effect is a phenomenon where the presence of a highly skilled coach significantly improves the performance of athletes. Due to the fact that coaches are not bound to a specific country, they can have a significant impact on the medal counts of multiple nations. Identifying the Great Coach Effect and quantifying its impact on medal counts is crucial for predicting Olympic success.

\subsection{Identification}

We reckon that the Great Coach Effect can be identified by comparing the performance of athletes under the same coach across different countries. It should, by nature, cause one country to have a sharp increase in their medal count in specific sports (otherwise this coach wouldn't be that great), and one other country to go through a gradual-to-sharp decrease in their medal count in the same sports. This is because the coach basically can only focus on coaching one country simultaneously, and the athletes from the other country would not receive the same level of training and support.

Again, due to the inherent limitation of usable dataset, we can only seek to identify the Great Coach Effect from this characteristic, rather than other methods such as relying on coach information to deduce the effect.

Therefore, we propose a two-step approach to identify and quantify the Great Coach Effect:

\begin{enumerate}
    \item Identify such pattern across the dataset.
    \item Refer to outside sources to interpret and cross-verify the result, finding out which specific coach is causing the effect each time.
\end{enumerate}





\subsection{Quantification}
This program is designed to identify the Great Coach Effect, a phenomenon where the presence of a highly skilled coach significantly enhances the performance of athletes in specific sports, often leading to a noticeable shift in medal counts between countries. Coaches, unlike athletes, are not bound to a single nation and may transfer their expertise, potentially causing one nation to experience a sharp increase in medals while another experiences a decline in the same sport. Identifying this effect can provide valuable insights for predicting Olympic success and optimizing investment in coaching.

Key Idea of the Pattern
The Great Coach Effect is characterized by two primary observations:

A sharp increase in medal counts for one country in a specific sport over a defined time period.
A gradual-to-sharp decrease in medal counts for another country in the same sport over a similar or overlapping time frame.
By focusing on these trends, the program aims to isolate and quantify the impact of influential coaches on national performances.

Methodology
The program uses a two-step approach to identify the pattern:

Data Analysis and Trend Detection

The dataset is grouped by country (NOC), sport, and year, with a weighted medal count (Gold = 3, Silver = 2, Bronze = 1) calculated for each combination.
For each country and sport, the program uses linear regression on the most recent 12 years of data to calculate the slope of the medal trend over time:
A positive slope exceeding a defined threshold (POS\_K\_LOWER) indicates a sharp increase.
A negative slope below another threshold (NEG\_K\_UPPER) indicates a sharp decrease.
Countries with significant trends are recorded as having possible increases or decreases.
Pattern Matching and Pairing

The program pairs countries with overlapping trends in the same sport:
The overlap period is checked to ensure it spans at least four years and does not exceed a defined gap (OVERLAP\_GAP\_UPPER).
Paired trends suggest the transfer of coaching expertise from one country to another.
Output
The program outputs three key results:

Possible Increases: Countries with significant positive slopes in their medal counts for specific sports.
Possible Decreases: Countries with significant negative slopes in the same metrics.
Paired Trends: Potential Great Coach Effect cases, listing both the increasing and decreasing countries, the sport, and the overlapping time frame.
This data-driven approach allows for systematic identification of the Great Coach Effect, which can then be cross-referenced with external sources to pinpoint the specific coaches responsible for these shifts. By doing so, the program provides actionable insights into the strategic importance of coaching in achieving Olympic success.




\subsection{Exemplification}
From the patterns generated by the Python program above, we can exclude obviously irrelevant ones and sift out some of the patterns that have causality with the Great Coach Effect. By analyzing the performance trends in specific sports, we identified notable cases where the transfer of a coach coincided with a sharp increase in performance of  one country and a decline or stagnation for another. Below are three exemplifications of the Great Coach Effect based on historical Olympic data.

\subsubsection{Synchronized Swimming Women Team: Ana Tarré}
Ana Tarré's influence as a coach is clearly reflected in the medal trends for synchronized swimming. During her tenure with Spain (1996–2012), the team gradually improved, earning higher medals, peaking with a Silver medal in 2008 and a Bronze in 2012. However, after she transitioned to coaching China in 2012, Spain had dropped out of the podium positions since then until 2024, while China experienced a gradual rise, achieving a series of Silver medals and finnaly a Gold medal in 2024. This shift strongly aligns with the Great Coach Effect, as her expertise and strategies appear to have propelled China's team to the top while Spain struggled without her guidance.

\subsubsection{Gymnastics Women Team All-Around: Béla Károlyi}
Béla Károlyi's legendary coaching career demonstrates a clear pattern of the Great Coach Effect. While coaching Romania (1974–1981), he elevated the team to global prominence, culminating in Silver medals in 1976 and 1980. After moving to coach the United States (1981–2016), Romania's performance declined, while the U.S. women's gymnastics team emerged as a dominant force, securing multiple Gold medals, including in 1996, 2012, and 2016. The sharp contrast in medal trends between the two nations during and after his coaching periods highlights his pivotal role in shaping team success.

\subsubsection{Volleyball Women Team: Lang Ping}
Lang Ping's coaching career also showcases a distinct pattern of influence across both China and the United States. During her first tenure with China (1995–2005), the team achieved notable successes, including a Gold medal in 2004. After transitioning to coach the U.S. team (2005–2013) which had won few medals in history, U.S. quickly surpassed China and won 2 consecutive Silver medals. When Lang Ping returned to coach China in 2013, the Chinese team quickly regained its dominance with a Gold in 2016. This fluctuation in performance underscores her unique ability to transform teams and highlights the strategic importance of securing top-tier coaching talent.

% These examples illustrate how the transfer of elite coaches can create significant shifts in medal performance across countries, substantiating the Great Coach Effect. The patterns not only validate the causality between coaching expertise and medal counts but also provide insights for national Olympic committees to consider investing in high-caliber coaching for sustained success.
\subsubsection{Cross Validation with Program-based Data}


\section{Sensitivity Analysis}

\subsection{Strengths}

\subsection{How to cite?}
bibliography cite use \cite{1}

% AI cite use \AIcite{AI1,AI2,AI3}

\begin{thebibliography}{99}
    
    \bibitem{1} Ball, D. W. (1972). Olympic Games competition: Structural correlates of national success. International Journal of Comparative Sociology, 15, 186–200.
    \bibitem{2} Bernard, A. B., \& Busse, M. R. (2004). Who wins the Olympic Games: Economic resources and medal totals. Review of Economics and Statistics, 86, 414–417.
    \bibitem{3} Bian, X. (2005). Predicting Olympic Medal Counts: The Effects of Economic Development on Olympic Performance. The Park Place Economist, 13, 37-44.
    \bibitem{4} Forrest, D., Sanz, I., \& Tena, J. D. (2010). Forecasting national team medal totals at the Summer Olympic Games. International Journal of Forecasting, 26, 576–588.
    \bibitem{5} Schlembach, C., Schmidt, S. L., Schreyer, D., \& Wunderlich, L. (2022). Forecasting the Olympic medal distribution – A socioeconomic machine learning model. Technological Forecasting \& Social Change, 175, 121314.
    \bibitem{6} Tcha, M., \& Pershin, V. (2003). Reconsidering performance at the Summer Olympics and revealed comparative advantage. Journal of Sports Economics, 4, 216–239.
    \bibitem{7} Bernard A B, Busse M R. Who wins the Olympic Games: Economic resources and medal totals[J]. Review of economics and statistics, 2004, 86(1): 413-417.

\end{thebibliography}

\begin{appendices}

\section{First appendix}

% In addition, your report must include a letter to the Chief Financial Officer (CFO) of the Goodgrant Foundation, Mr. Alpha Chiang, that describes the optimal investment strategy, your modeling approach and major results, and a brief discussion of your proposed concept of a return-on-investment (ROI). This letter should be no more than two pages in length.

\begin{letter}{Dear, Mr. Alpha Chiang}

    \vspace{\parskip}

    Sincerely yours,

    Your friends

\end{letter}

\section{Second appendix}

    % \lstinputlisting[language=C++]{./code/mcmthesis-sudoku.cpp}

\end{appendices}

% AI report begins here
\AImatter
\begin{ReportAiUse}{9}
\bibitem{AI1}



\end{ReportAiUse}

\end{document}
